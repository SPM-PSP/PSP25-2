\documentclass[12pt]{article}
\usepackage{ctex}
\usepackage{amsmath}
\usepackage{array}
\usepackage{booktabs}

\begin{document}

\textbf{8. 今有10组观察数据由下表给出:}

\begin{center}
\begin{tabular}{|c|c|c|c|c|c|c|c|c|c|c|}
\hline
$x$ & 0.5 & $-0.8$ & 0.9 & $-2.8$ & 6.5 & 2.3 & 1.6 & 5.1 & $-1.9$ & $-1.5$ \\
\hline
$y$ & $-0.3$ & $-1.2$ & 1.1 & $-3.5$ & 4.6 & 1.8 & 0.5 & 3.8 & $-2.8$ & 0.5 \\
\hline
\end{tabular}
\end{center}

应用线性模型 $y_i = \beta_0 + \beta_1 x_i + \varepsilon_i$,$i = 1,2,\cdots,10$,假定诸 $\varepsilon_i$ 相互独立,且均服从分布 $N(0,\sigma^2)$。

(1) 求 $\beta_0, \beta_1$ 的最小二乘估计。

(2) 计算剩余方差 $\sigma_e^2$。

(3) 在显著性水平 $\alpha = 0.05$ 下检验假设 $H_0: \beta_1 = 0$。

(4) 求 $y$ 的置信水平为 0.95 的预测区间。

\vspace{1em}

\textbf{11. 设 $y_i = \beta_0 + \beta_1 x_i + \varepsilon_i$ ($i = 1, 2, \cdots, n$),其中诸 $\varepsilon_i$ 相互独立,且均服从分布 $N(0, \sigma^2)$。}

(1) 导出假设 $H_0: \beta_1 = 0$ 的 $F$ 检验统计量。

(2) 如果 $\bar{x} = 0$,导出假设 $H_0: \beta_0 = \beta_1$ 的 $F$ 检验统计量。

\vspace{1em}

\textbf{14. 设 $Y = X\boldsymbol{\beta} + \boldsymbol{\varepsilon}$,$E(\boldsymbol{\varepsilon}) = \mathbf{0}$,$\text{Var}(\boldsymbol{\varepsilon}) = \sigma^2 \mathbf{M}$,$\mathbf{M}$ 为已知的正定矩阵,$X$ 为 $m \times k$ 阶矩阵,试证 $\hat{\boldsymbol{\beta}} = (\mathbf{X}^T\mathbf{M}^{-1}\mathbf{X})^{-1}\mathbf{X}^T\mathbf{M}^{-1}Y$ 使 $(Y - X\boldsymbol{\beta})^T\mathbf{M}^{-1}(Y - X\boldsymbol{\beta})$ 达到极小,称 $\hat{\boldsymbol{\beta}}$ 为 $\boldsymbol{\beta}$ 的加权最小二乘估计(提示:令 $z = \mathbf{K}^{-1}Y$,$\mathbf{X}^* = \mathbf{K}^{-1}\mathbf{X}$,其中 $\mathbf{M} = \mathbf{K}\mathbf{K}^T$)。}

\end{document}
